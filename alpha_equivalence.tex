
\documentclass{article}
\usepackage{geometry}
\geometry{a4paper, margin=1in}
\usepackage{booktabs}
\begin{document}
\title{Alpha-Equivalence Analysis via De Bruijn Indices}
\author{Antigravity Agent}
\date{\today}
\maketitle

\section{Methodology}
We applied lambda calculus-inspired techniques to detect Common Subexpressions (CSE):
\begin{itemize}
    \item \textbf{De Bruijn Indices}: Variable names normalized to binding distance
    \item \textbf{Hash Consing}: DAG canonicalization for structural equality
    \item \textbf{Tree Pattern Matching}: Subtree extraction and comparison
\end{itemize}

\section{Common Patterns Detected}
The following patterns (modulo alpha-equivalence) appear in multiple locations:

\begin{table}[h]
\centering
\begin{tabular}{lc}
\toprule
\textbf{Pattern Hash} & \textbf{Occurrences} \\
\midrule
43425b64ca35... & 30 \\
292075a08b38... & 33 \\
f06b4bafd3e6... & 9 \\
7ce493cf3e90... & 12 \\
598af978dc54... & 10 \\
dc2c6c712030... & 7 \\
f419212f408f... & 10 \\
1808aa08862d... & 8 \\
19244e041e5c... & 8 \\
8bdc70be4782... & 8 \\
a4c390856931... & 8 \\
14317903449f... & 7 \\
fab5e31572d4... & 7 \\
6694aeeeee47... & 5 \\
9ea622762d11... & 6 \\

\bottomrule
\end{tabular}
\end{table}

\section{Interpretation}
Patterns with high occurrence counts indicate opportunities for refactoring via Common Subexpression Elimination (CSE). The use of De Bruijn indices ensures we detect equivalences even when variable names differ.

\end{document}
