\documentclass{article}
\usepackage[utf8]{inputenc}
\usepackage{geometry}
\geometry{a4paper, margin=1in}
\usepackage{listings}
\usepackage{xcolor}

\title{Summary of Spip Cleanup Feature Implementation}
\author{Gemini CLI Agent}
\date{January 30, 2026}

\begin{document}

\maketitle

\section{Overview}
A new cleanup and maintenance feature has been added to the \texttt{spip} CLI tool. This feature addresses the need for managing disk space, removing orphaned environments, and optimizing the underlying Git repositories.

\section{New Features}
\begin{itemize}
    \item \textbf{New Commands:} Added \texttt{spip gc} as a primary command for garbage collection, also available via the \texttt{spip cleanup} alias. Added support for \texttt{--all} to remove all managed environments.
    \item \textbf{Disk Usage Statistics:} The tool now reports detailed disk usage before and after cleanup, including sizes for the Git repository, environments, and the package database.
    \item \textbf{Orphan Pruning:} Automatically identifies and removes environments for projects that no longer exist on the filesystem or have been unused for more than 30 days.
    \item \textbf{Temporary File Cleanup:} Cleans up \texttt{.whl} files, \texttt{temp\_venv\_} directories, and other temporary artifacts in the \texttt{.spip} root.
    \item \textbf{Script Maintenance:} Removes unrecognized or obsolete Python scripts from the \texttt{.spip/scripts} directory.
    \item \textbf{Git Optimization:} Executes \texttt{git gc --prune=now --aggressive} on both the main environment repository and the package metadata database. This is rate-limited to once every 24 hours to prevent excessive system load.
\end{itemize}

\section{Implementation Details}
The implementation was performed in C++23 within \texttt{spip.cpp}. Key functions added or modified include:
\begin{itemize}
    \item \texttt{show\_usage\_stats(const Config\& cfg)}: Calculates and prints directory sizes using \texttt{std::filesystem}.
    \item \texttt{cleanup\_spip(Config\& cfg)}: Orchestrates the multi-stage cleanup process.
    \item Updated \texttt{run\_command} router to handle the new \texttt{gc} command and updated the \texttt{list} command to use detailed stats.
\end{itemize}

\section{Verification}
The feature was compiled using \texttt{g++ -std=c++23} and verified on the local system. The initial run successfully reclaimed approximately 4GB of disk space by pruning orphaned branches and compacting the Git repository.

\end{document}